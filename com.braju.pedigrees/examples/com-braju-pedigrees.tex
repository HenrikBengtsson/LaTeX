%%%%%%%%%%%%%%%%%%%%%%%%%%%%%%%%%%%%%%%%%%%%%%%%%%%%%%%%%%%%%%%%%%%%%%
% Examples for the com.braju.pedigrees package
% 
% Henrik Bengtsson, hb@maths.lth.se, February 2004
%
% Requirements:
% [1] Henrik Bengtsson, com.braju.pedigrees.sty, 2004, 
%     http://www.maths.lth.se/~hb/mypackages/
% [2] Timothy Van Zandt, fancyvrb, July 1998
%     http://www.ctan.org/tex-archive/macros/latex/contrib/fancyvrb/
% [3] Timothy van Zandt, PSTricks, 1997
%     http://www.tug.org/applications/PSTricks/, 
%     http://www.pstricks.de/
%%%%%%%%%%%%%%%%%%%%%%%%%%%%%%%%%%%%%%%%%%%%%%%%%%%%%%%%%%%%%%%%%%%%%%
\documentclass[a4paper,dvips]{article}
\usepackage{com.braju.pedigrees}   % If not available, see [1].
\usepackage{fancyvrb}              % If not available, see [2].
\usepackage{graphicx}

\catcode`\@=11%

\newcommand{\example}[1]{%
 \subsection{Example #1}
 \begin{center}
  \input{figures/#1}
 \end{center}
 \VerbatimInput[fontsize=\small,numbers=left,obeytabs=true,frame=lines,commentchar=\%]{figures/#1}
}

\title{Examples how to use the com.braju.pedigrees package}
\author{Henrik Bengtsson, \texttt{hb@maths.lth.se}}
\date{\today}

\begin{document}

\maketitle

\tableofcontents

\section{Examples}
\example{figure001.tex}
\example{figure002.tex}
\example{figure003.tex}
\example{figure004.tex}
\example{figure005.tex}
\example{figure006.tex}
\example{figure007.tex}
\example{figure008.tex}
\example{figure009.tex}
\example{figure010.tex}
\example{figure011.tex}
\example{figure012.tex}
\example{figure013.tex}

\newpage
\appendix
\section{Source of com.braju.pedigrees.sty}
 \VerbatimInput[fontsize=\small,numbers=left,obeytabs=true,frame=lines]{com.braju.pedigrees.sty}

\section{Source of com.braju.pstricks.sty}
 \VerbatimInput[fontsize=\small,numbers=left,obeytabs=true,frame=lines]{com.braju.pstricks.sty}

\end{document}


%%%%%%%%%%%%%%%%%%%%%%%%%%%%%%%%%%%%%%%%%%%%%%%%%%%%%%%%%%%%%%%%%%%%%%
% HISTORY
% 2004-02-06
% o Created. /HB
%%%%%%%%%%%%%%%%%%%%%%%%%%%%%%%%%%%%%%%%%%%%%%%%%%%%%%%%%%%%%%%%%%%%%%
